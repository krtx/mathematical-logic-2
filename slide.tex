\documentclass[dvipdfmx,11pt]{beamer}

\usepackage{bm}
\usepackage[figurename=,tablename=]{caption}
\usepackage{multicol}
\usepackage{bcprules}
\usepackage{multirow}
\usepackage{colortbl}
\usepackage{stmaryrd}
\usepackage{mathtools}
\usepackage{mathabx}
\usepackage[yyyymmdd]{datetime}
\usepackage{url}

\DeclareFontFamily{U}{mathc}{}
\DeclareFontShape{U}{mathc}{m}{it}%
{<->s*[1.03] mathc10}{}

\DeclareMathAlphabet{\mathscr}{U}{mathc}{m}{it}

\DeclareMathOperator{\Pow}{\mathscr{Pow}}

\newcommand\DEF{\overset{\mathit{def}}{\equiv}}

\usepackage{catchfilebetweentags}

\setbeamertemplate{navigation symbols}{}
\setbeamertemplate{footline}[frame number]{}
\setbeamerfont{footline}{size=\small}
\setbeamerfont{date}{size=\footnotesize}

% Use the same font in nested itemizes.
\setbeamerfont*{itemize/enumerate body}{}
\setbeamerfont*{itemize/enumerate subbody}{parent=itemize/enumerate body}
\setbeamerfont*{itemize/enumerate subsubbody}{parent=itemize/enumerate body}

% Change font
\renewcommand{\kanjifamilydefault}{\gtdefault}
\renewcommand{\dateseparator}{-}

\newcommand{\predl}{\textcircled{\scriptsize 述}}
\newcommand{\propl}{\textcircled{\scriptsize 命}}

\newtheorem{proposition}[theorem]{Proposition}
\newtheorem{defi}[theorem]{定義}
\newtheorem{theo}[theorem]{定理}

%% \input{local}

\title{数理論理学}
\subtitle{5.4 推論}
\author{樹下}
\date{\today}

\begin{document}

\maketitle

\begin{frame}{目次}
  \begin{itemize}

  \item 5.4.1 推論の妥当性
    \begin{itemize}
    \item \predl{}における意味論的含意(推論)の定義
    \end{itemize}
  \item 5.4.2 \propl{}における妥当な推論
    \begin{itemize}
    \item \propl{}における妥当な推論は,\predl{}においても妥当
    \end{itemize}

  \item 5.4.3 全称・存在量化の導入律と除去律
  \item 5.4.4 \predl{}の推論
  \item 5.4.5 \(\alpha\)同値性
  \item 5.4.6 置き換え

    \begin{itemize}
    \item 5.4.3--5.4.6 は,全称・存在量化に焦点を置いた,\predl{}における命題の同値性についての話
    \end{itemize}

  \end{itemize}
\end{frame}

\begin{frame}{5.4.1 推論の妥当性}
  \predl{}での話

  \begin{defi}[5.66 意味論的含意]
    \(\Gamma\),\(\Delta\)を論理式列(ただし、\(\left|\Delta\right|\)は高々1)とする.
    \(\Gamma\)が\(\Delta\)を意味論的に含意する(\(\Gamma\vDash\Delta\)と記す)とは,
    \(\Gamma\)に含まれるすべての論理式\(\phi\)について\(\llbracket\phi\rrbracket_{M,g}=1\)
    であり,かつ\(\Delta\)に含まれるすべての論理式\(\psi\)について
    \(\llbracket\psi\rrbracket_{M,g}=0\)であるような解釈\(\langle M,g\rangle\)が存在しない,
    ということである.\(\Gamma\vDash\Delta\)ではないとき,\(\Gamma\nvDash\Delta\)と記す.
  \end{defi}

  \begin{defi}[5.68 充足性]
    \(\llbracket\phi\rrbracket_{M,g}=1\)であるとき,解釈\(\langle M,g \rangle\)は
    論理式\(\phi\)を充足するという.解釈\(\langle M,g \rangle\)が論理式\(\Gamma\)に属する
    すべての論理式を充足するとき,\(\langle M,g \rangle\)は\(\Gamma\)を同時に充足するといい,
    \(\langle M,g \rangle\vDash\Gamma\)と記す.
  \end{defi}
\end{frame}

\begin{frame}{5.4.1 推論の妥当性}
  \begin{defi}[3.56 充足可能性]
    論理式列\(\Gamma\)を同時に充足する解釈が存在することを,\(\Gamma\)は充足可能である,という.
    論理式列\(\Gamma\)が充足可能ではないことを,\(\Gamma\)は充足不能である,という.
  \end{defi}

  \begin{theo}[3.57]
    \(\Gamma\),\(\Delta\)を論理式列とすると,以下が成り立つ.
    \[
    \Gamma\vDash\Delta \iff \neg\Delta,\Gamma\textit{が充足不能である.}
    \]
  \end{theo}

  定理3.57は一階述語論理においても成立する.
\end{frame}

\begin{frame}{5.4.2 \propl{}における妥当な推論}
  \begin{theo}[5.69]
    \propl{}の恒真式に現れる命題記号を,\predl{}の論理式に置き換えた論理式は,\predl{}の恒真式である.
  \end{theo}

  \begin{example}
    \begin{itemize}
    \item \(P\rightarrow (\neg P\rightarrow Q)\)
    \item \(x=a\rightarrow (\neg (x=a)\rightarrow (y>b))\)
    \end{itemize}
  \end{example}

  ※同じ命題記号は同じ論理式で置き換える
\end{frame}

\begin{frame}{5.4.2 \propl{}における妥当な推論}
  \begin{block}{証明}
  \begin{itemize}
  \item ある\propl{}の論理式\(\Phi\)に対応する\predl{}の論理式を\(\Phi'\)とする
  \item \predl{}の任意の解釈\(\langle M,g \rangle\)は\propl{}の解釈とみなすことができる
    \begin{itemize}
    \item \(\Phi\)中の各命題記号に対応する\predl{}の論理式の,解釈\(\langle M,g \rangle\)のもとでの値は\(1\)か\(0\)
    \end{itemize}
  \item つまり,ある\(\langle M,g \rangle\)に対応する\(I\)があって
    \(\llbracket\Phi'\rrbracket_{M,g}=\llbracket\Phi\rrbracket_{I}\)
    \begin{itemize}
    \item 注意として,復号論理式と真理関数の解釈は,\propl{}と\predl{}で共通している
    \end{itemize}
  \item 任意の\(I\)のもとで\(\llbracket\Phi\rrbracket_{I} = 1\)なので\(\llbracket\Phi'\rrbracket_{M,g} = 1\)
  \end{itemize}
  \end{block}
\end{frame}

\begin{frame}{5.4.2 \propl{}における妥当な推論}
  \begin{itemize}
  \item 3.2.4で列挙した\propl{}の恒真式は,それぞれ\predl{}においても恒真式
  \item 同様のことが推論についても成り立つ
  \end{itemize}

  \begin{theo}[5.70]
    \propl{}の妥当な推論に現れる命題記号を,\predl{}の論理式に置き換えた推論は,\predl{}の妥当な推論である.
  \end{theo}

  \begin{block}{証明}
    定理~5.69と同様.
  \end{block}

  \mbox{}

  ある体系で真な主張を保存する拡張を保守的な拡張と言った覚え・・・
\end{frame}

\begin{frame}{5.4.3 全称・存在量化の導入律と除去律}
  \fontsize{9pt}{7.2}\selectfont
  \begin{theo}[5.71 全称除去律]
    任意の論理式\(\phi,\psi\),項\(\tau\),変項\(\xi\)について以下が成り立つ.
    \[
    \phi [ \tau / \xi ] \vDash \psi \ \Longrightarrow \ \forall \xi \phi \vDash \psi
    \]
  \end{theo}
  \begin{theo}[5.72 存在導入律]
    任意の論理式の列\(\Gamma\),論理式\(\phi\),項\(\tau\),変項\(\xi\)について以下が成り立つ.
    \[
    \Gamma \vDash \psi [ \tau / \xi ] \ \Longrightarrow \ \Gamma \vDash \exists \xi \phi
    \]
  \end{theo}
  \begin{theo}[5.73 全称導入律]
    任意の論理式の列\(\Gamma\),論理式\(\phi\),変項\(\xi,\zeta\)(ただし\(\zeta \notin \textit{fv}(\Gamma) \cup \textit{fv}(\forall \xi \phi)\))について以下が成り立つ.
    \[
    \Gamma \vDash \phi [ \zeta / \xi ] \ \Longrightarrow \ \Gamma \vDash \forall \xi \phi
    \]
  \end{theo}
  \begin{theo}[5.74 存在除去律]
    任意の論理式\(\phi,\psi\),変項\(\xi,\zeta\)(ただし\(\zeta \notin \textit{fv}(\exists \xi \phi) \cup \textit{fv}(\psi)\))について以下が成り立つ
    \[
    \phi [ \zeta / \xi ] \vDash \psi \ \Longrightarrow \ \exists \xi \phi \vDash \psi
    \]
  \end{theo}
\end{frame}

\begin{frame}{5.4.3 全称・存在量化の導入律と除去律}
  \begin{theo}[5.71 全称除去律]
    任意の論理式\(\phi,\psi\),項\(\tau\),変項\(\xi\)について以下が成り立つ.
    \[
    \phi [ \tau / \xi ] \vDash \psi \ \Longrightarrow \ \forall \xi \phi \vDash \psi
    \]
  \end{theo}

  \begin{block}{証明}
    \(\phi [ \tau / \xi ] \vDash \psi\)を前提として任意の\(\langle M,g \rangle\)について\(\langle M,g \rangle \vDash \forall \xi \phi\)ならば\(\llbracket \psi \rrbracket_{M,g}=1\)であることを示せばよい.
    \[
    \begin{array}{rll}
      & \langle M,g \rangle \vDash \forall \xi \phi & \\
      \iff & \llbracket \forall \xi \phi \rrbracket_{M,g} = 1 & \\
      \iff & \textit{すべての}a\in D_{M}\textit{について}\llbracket \phi \rrbracket_{M,g[\xi\mapsto a]} = 1 & \\
      \Longrightarrow & \llbracket \phi \rrbracket_{M,g[\xi \mapsto \llbracket \tau \rrbracket_{M,g}]} = 1 & (\llbracket \tau \rrbracket_{M,g} \in D_{M}) \\
      \iff & \llbracket \phi [ \tau / \xi ] \rrbracket_{M,g} = 1 & (\textit{補題}5.61(2)) \\
      \Longrightarrow & \llbracket \psi \rrbracket_{M,g} = 1 & (\textit{前提})
    \end{array}
    \]
  \end{block}
\end{frame}

\begin{frame}{5.4.3 全称・存在量化の導入律と除去律}
  ここで,以下の補題をよく使うので,ホワイトボードに書く

  \begin{itemize}
  \item 補題 5.42 (1)
    \[
    \phi[\xi/\xi]\equiv\phi
    \]
  \item 補題 5.56
    \[
    g[\xi\mapsto a][\zeta\mapsto b]=g[\zeta\mapsto b][\xi\mapsto a] \ (\xi\not\equiv\zeta)
    \]
  \item 補題 5.59 (2)
    \[
    \xi\not\in\textit{fv}(\phi)\Longrightarrow\llbracket\phi\rrbracket_{M,g}=\llbracket\phi\rrbracket_{M,g[\xi\mapsto a]}
    \]
  \item 補題 5.61 (2)
    \[
    \llbracket\phi[\tau/\xi]\rrbracket_{M,g}=\llbracket\phi\rrbracket_{M,g[\xi\mapsto\llbracket\tau\rrbracket_{M,g}]}
    \]
  \end{itemize}
\end{frame}

\begin{frame}{5.4.3 全称・存在量化の導入律と除去律}
  \begin{theo}[5.72 存在導入律]
    任意の論理式の列\(\Gamma\),論理式\(\phi\),項\(\tau\),変項\(\xi\)について以下が成り立つ.
    \[
    \Gamma \vDash \psi [ \tau / \xi ] \ \Longrightarrow \ \Gamma \vDash \exists \xi \phi
    \]
  \end{theo}

  \begin{block}{証明}
    \(\Gamma \vDash \psi [ \tau / \xi ]\)を前提として,任意の\(\langle M,g \rangle\)について,
    \(\langle M,g \rangle \vDash \Gamma\)ならば\(\llbracket \exists \xi \phi \rrbracket_{M,g}=1\)
    であることを示せばよい.\(\tau\)を解釈した\(D_M\)の値を\(\xi\)の値として使える
    \[
    \begin{array}{rll}
      & \langle M,g \rangle \vDash \Gamma & \\
      \Longrightarrow & \llbracket \phi [ \tau / \xi ] \rrbracket_{M,g} = 1 & (\textit{前提}) \\
      \iff & \llbracket \phi \rrbracket_{M,g[\xi\mapsto\llbracket\tau\rrbracket_{M,g}]} = 1 & (\textit{補題5.61(2)}) \\
      \Longrightarrow & \exists a \in D_M \ s.t. \ \llbracket \phi \rrbracket_{M,g[\xi\mapsto a]} = 1 & \\
      \iff & \llbracket \exists \xi \phi \rrbracket_{M,g} = 1 &
    \end{array}
    \]
  \end{block}
\end{frame}

\begin{frame}{5.4.3 全称・存在量化の導入律と除去律}
  \begin{theo}[5.73 全称導入律]
    任意の論理式の列\(\Gamma\),論理式\(\phi\),変項\(\xi,\zeta\)(ただし\(\zeta \notin \textit{fv}(\Gamma) \cup \textit{fv}(\forall \xi \phi)\))について以下が成り立つ.
    \[
    \Gamma \vDash \phi [ \zeta / \xi ] \ \Longrightarrow \ \Gamma \vDash \forall \xi \phi
    \]
  \end{theo}
  \begin{block}{証明}
    \(\Gamma \vDash \phi [ \zeta / \xi ]\)を前提として,任意の\(\langle M,g \rangle\)について,
    \(\langle M,g \rangle \vDash \Gamma\)ならば\(\llbracket \forall \xi \phi \rrbracket_{M,g}=1\)
    であることを示せばよい.\(\xi\not\equiv\zeta\)と\(\xi\equiv\zeta\)で場合分けする.
  \end{block}
\end{frame}

\begin{frame}{5.4.3 全称・存在量化の導入律と除去律}
  \(\xi\not\equiv\zeta\)の場合\\\mbox{}\\

  \(\zeta\notin\textit{fv}(\forall\xi\phi)=\textit{fv}(\phi)-\{\xi\}\)と\(\xi\not\equiv\zeta\)
  より,\(\zeta\notin\textit{fv}(\phi)\)である.任意の\(a(\in D_M)\)について
  \[
  \begin{array}{rll}
    & \langle M,g \rangle\vDash\Gamma & \\
    \Longrightarrow & \langle M,g[\zeta\mapsto a]\rangle\vDash\Gamma & (\zeta\notin\textit{fv}(\Gamma), \textit{補題5.59(2)}) \\
    \Longrightarrow & \llbracket\phi[\zeta/\xi]\rrbracket_{M,g[\zeta\mapsto a]} \vDash \Gamma & (\textit{前提}(\Gamma \vDash \phi [ \zeta / \xi ]))\\
    \iff & \llbracket\phi\rrbracket_{M,g[\zeta\mapsto a][\xi\mapsto \llbracket\zeta\rrbracket_{M,g[\zeta\mapsto a]}]} = 1 & (\textit{補題5.61(2)})\\
    \iff & \llbracket\phi\rrbracket_{M,g[\zeta\mapsto a][\xi\mapsto a]} = 1 & \\
    \iff & \llbracket\phi\rrbracket_{M,g[\xi\mapsto a][\zeta\mapsto a]} = 1 & (\zeta\not\equiv\xi,\textit{補題5.56}) \\
    \Longrightarrow & \llbracket\phi\rrbracket_{M,g[\xi\mapsto a]} = 1 & (\zeta\notin\textit{fv}(\phi),\textit{補題5.59(2)}) \\
    \iff & \llbracket\forall\xi\phi\rrbracket_{M,g} = 1 & \\
  \end{array}
  \]
\end{frame}

\begin{frame}{5.4.3 全称・存在量化の導入律と除去律}
  \(\xi\equiv\zeta\)の場合\\\mbox{}\\

  任意の\(a\in D_M\)について

  \[
  \begin{array}{rll}
    & \langle M,g \rangle \vDash \Gamma & \\
    \Longrightarrow & \langle M,g[\zeta\mapsto a] \rangle & (\zeta\notin\textit{fv}(\Gamma), \textit{補題5.59(2)}) \\
    \iff & \langle M,g[\xi\mapsto a] \rangle & (\zeta\equiv\xi) \\
    \Longrightarrow & \llbracket \phi \rrbracket_{M,g[\xi\mapsto a]} = 1 & (\textit{前提}) \\
    \iff & \llbracket\forall\xi\phi\rrbracket_{M,g} = 1 & \\
  \end{array}
  \]
\end{frame}

\begin{frame}{5.4.3 全称・存在量化の導入律と除去律}
  \begin{theo}[5.74 存在除去律]
    任意の論理式\(\phi,\psi\),変項\(\xi,\zeta\)(ただし\(\zeta \notin \textit{fv}(\exists \xi \phi) \cup \textit{fv}(\psi)\))について以下が成り立つ
    \[
    \phi [ \zeta / \xi ] \vDash \psi \ \Longrightarrow \ \exists \xi \phi \vDash \psi
    \]
  \end{theo}

  \begin{block}{証明}
    \(\phi[\zeta/\xi]\vDash\psi\)を前提として,任意の\(\langle M,g \rangle\)について,
    \(\llbracket\psi\rrbracket_{M,g}=0\)ならば\(\langle M,g \rangle\nvDash\exists\xi\phi\)
    であることを示せばよい(対偶より).
    \(\xi\not\equiv\zeta\)と\(\xi\equiv\zeta\)で場合分けする.
  \end{block}
\end{frame}

\begin{frame}{5.4.3 全称・存在量化の導入律と除去律}
  \(\xi\not\equiv\zeta\)の場合\\\mbox{}\\

  \(\zeta\not\in\textit{fv}(\exists\xi\phi)=\textit{fv}(\phi)-\{\xi\}\)より,
  \(\zeta\not\in\textit{fv}(\phi)\)である.任意の\(a(\in D_M)\)について

  \[
  \begin{array}{rll}
    & \llbracket\psi\rrbracket_{M,g}=0 & \\
    \Longrightarrow & \llbracket\psi\rrbracket_{M,g[\zeta\mapsto a]}=0 & (\zeta\not\in\textit{fv}(\psi),\textit{補題5.59(2)}) \\
    \Longrightarrow & \llbracket\phi[\zeta/\xi]\rrbracket_{M,g[\zeta\mapsto a]}=0 & (\textit{前提の対偶}) \\
    \iff & \llbracket\phi\rrbracket_{M,g[\zeta\mapsto a][\xi\mapsto\llbracket\zeta\rrbracket_{M,g[\zeta\mapsto a]}]}=0 & (\textit{補題5.61(2)}) \\
    \iff & \llbracket\phi\rrbracket_{M,g[\zeta\mapsto a][\xi\mapsto a]}=0 & \\
    \iff & \llbracket\phi\rrbracket_{M,g[\xi\mapsto a][\zeta\mapsto a]}=0 & (\zeta\not\equiv,\textit{補題5.56}) \\
    \iff & \llbracket\phi\rrbracket_{M,g[\xi\mapsto a]}=0 & (\zeta\not\in\textit{fv}(\phi),\textit{補題5.59(2)}) \\
    \iff & \llbracket\exists\xi\phi\rrbracket_{M,g}=0 & 
  \end{array}
  \]
\end{frame}

\begin{frame}{5.4.3 全称・存在量化の導入律と除去律}
  前提

  \[
  \phi[\zeta/\xi]\vDash\psi \iff \forall \langle M,g \rangle \ \llbracket\phi[\zeta/\xi]\rrbracket_{M,g} = 1 \ \textit{ならば}\ \llbracket\psi\rrbracket_{M,g} \neq 0
  \]

  の対偶

  \[
  \forall \langle M,g \rangle \ \llbracket\psi\rrbracket_{M,g} = 0 \ \textit{ならば}\ \llbracket\phi[\zeta/\xi]\rrbracket_{M,g}\neq 1
  \]
\end{frame}

\begin{frame}{5.4.3 全称・存在量化の導入律と除去律}
  \(\xi\equiv\zeta\)の場合\\\mbox{}\\

  任意の\(a(\in D_M)\)について
  \[
  \begin{array}{rll}
    & \llbracket\psi\rrbracket_{M,g}=0 & \\
    \Longrightarrow & \llbracket\psi\rrbracket_{M,g[\zeta\mapsto a]}=0 & (\zeta\not\in\textit{fv}(\psi),\textit{補題5.59(2)}) \\
    \iff & \llbracket\psi\rrbracket_{M,g[\xi\mapsto a]}=0 & (\zeta\equiv\xi) \\
    \Longrightarrow & \llbracket\phi[\zeta/\xi]\rrbracket_{M,g[\xi\mapsto a]}=0 & (\textit{前提の対偶}) \\
    \iff & \llbracket\phi\rrbracket_{M,g[\xi\mapsto a]}=0 & (\zeta\equiv\xi,(\textit{定理5.42(1)})) \\
    \iff & \llbracket\exists\xi\phi\rrbracket_{M,g}=0 &
  \end{array}
  \]

\end{frame}

\begin{frame}{5.4.3 全称・存在量化の導入律と除去律}
  \begin{block}{解説5.75}
    全称導入律の但し書きを無視して間違った推論を導出する例

    \begin{description}
    \item [推論] \(\forall x \exists y F(x,y) \vDash \exists y \forall x F(x,y) \)
    \item [過程]
      \[
      \begin{array}{rrcll}
        (1) & \exists y F(x,y) & \vDash & \exists y F(x,y) & (\textit{同一律}) \\
        (2) & \forall x \exists y F(x,y) & \vDash & \exists y F(x,y) & (\textit{(1)と全称除去律}) \\
        (3) & F(x,y) & \vDash & F(x,y) & (\textit{同一律}) \\
        (4) & F(x,y) & \vDash & \forall x F(x,y) & (\textit{(3)と全称導入律}) \\
        (5) & \forall x F(x,y) & \vDash & \forall x F(x,y) & (\textit{同一律}) \\
        (6) & \forall x F(x,y) & \vDash & \exists y \forall x F(x,y) & (\textit{(5)と存在導入律}) \\
        (7) & F(x,y) & \vDash & \exists y \forall x F(x,y) & (\textit{(4),(6)とカット}) \\
        (8) & \exists y F(x,y) & \vDash & \exists y \forall x F(x,y) & (\textit{(7)と存在除去律}) \\
        (9) & \forall x \exists y F(x,y) & \vDash & \exists y \forall x F(x,y) & (\textit{(2),(8)とカット}) \\
      \end{array}
      \]
    \end{description}
  \end{block}
\end{frame}

\begin{frame}{5.4.3 全称・存在量化の導入律と除去律}
  間違ってるところ
  \[
  \begin{array}{rrcll}
    (3) & F(x,y) & \vDash & F(x,y) & (\textit{同一律}) \\
    (4) & F(x,y) & \vDash & \forall x F(x,y) & (\textit{(3)と全称導入律})
  \end{array}
  \]
  全称導入律において
  \[
  \begin{array}{rcl}
    \Gamma & := & F(x,y) \\
    \phi & := & F(x,y) \\
    \xi & := & x \\
    \zeta & := & x
  \end{array}
  \]
  とすれば導出されるが,\(x \in \textit{fv}(\Gamma) = \textit{fv}(F(x,y))\)なので但し書きを満たしていない
\end{frame}

\begin{frame}{5.4.3 全称・存在量化の導入律と除去律}
  \begin{block}{解説5.76}
    存在除去律の但し書きを無視して間違った推論を導出する例

    \begin{description}
    \item [推論] \(\exists x F(x) \vDash \forall x F(x) \)
    \item [過程]
      \[
      \begin{array}{rrcll}
        (1) & F(x) & \vDash & F(x) & (\textit{同一律}) \\
        (2) & \exists x F(x) & \vDash & F(x) & (\textit{(1)と存在除去律}) \\
        (3) & \exists x F(x) & \vDash & \forall x F(x) & (\textit{(2)と全称導入律}) \\
      \end{array}
      \]
    \end{description}

    (2)は存在除去律で\(\psi \ := \ F(x), \phi \ := \ F(x), \zeta \ := \ x, \xi \ := \ x\)
    とすれば導出されるが,\(x\in\textit{fv}(F(x))\)なので但し書きを満たしていない
  \end{block}
\end{frame}

\begin{frame}{5.4.3 全称・存在量化の導入律と除去律}
  \begin{block}{解説5.76}
    存在除去律の但し書きを無視して間違った推論を導出する例,別の過程

    \begin{description}
    \item [推論] \(\exists x F(x) \vDash \forall x F(x) \)
    \item [過程]
      \[
      \begin{array}{rrcll}
        (1) & F(x) & \vDash & F(x) & (\textit{同一律}) \\
        (2) & F(x) & \vDash & \forall x F(x) & (\textit{(1)と全称導入律}) \\
        (3) & \exists x F(x) & \vDash & \forall x F(x) & (\textit{(2)と存在除去律}) \\
      \end{array}
      \]
    \end{description}

    (2)が全称導入律の但し書きを満たしていない(\(x\in\textit{fv}(\Gamma)=\textit{fv}(F(x))\))
  \end{block}
\end{frame}

\begin{frame}{5.4.4 一階述語論理の推論}
  \begin{theo}[5.77 全称例化律]
    \[
    \forall \xi \phi \vDash \phi [ \tau / \xi ]
    \]
  \end{theo}
  すべてのものがある属性を持つ,という一般的な言明を前提として,具体例への言明が帰結される推論
  \begin{theo}[5.79 存在汎化律]
    \[
    \phi[\tau/\xi]\vDash\exists \xi \phi
    \]
  \end{theo}
  ある属性を持つ具体例を前提として,その属性を持つものが存在する,という言明が帰結される推論
\end{frame}

\begin{frame}{5.4.4 一階述語論理の推論}
  \begin{theo}[5.77 全称例化律]
    \[
    \forall \xi \phi \vDash \phi [ \tau / \xi ]
    \]
  \end{theo}

  \begin{block}{証明}
    \[
    \begin{array}{rrcll}
      (1) & \phi[\tau/\xi] & \vDash & \phi[\tau/\xi] & (\textit{同一律}) \\
      (2) & \forall \xi \phi & \vDash & \phi[\tau/\xi] & (\textit{(1)と全称除去律}) \\
    \end{array}
    \]
  \end{block}
\end{frame}

\begin{frame}{5.4.4 一階述語論理の推論}
  \begin{theo}[5.79 存在汎化律]
    \[
    \phi[\tau/\xi]\vDash\exists \xi \phi
    \]
  \end{theo}

  \begin{block}{証明}
    \[
    \begin{array}{rrcll}
      (1) & \phi [ \tau / \xi ] & \vDash & \phi [ \tau / \xi ] & (\textit{同一律}) \\
      (2) & \phi [ \tau / \xi ] & \vDash & \exists \xi \phi & (\textit{(1)と存在導入律}) \\
    \end{array}
    \]
  \end{block}
\end{frame}

\begin{frame}{5.4.4 一階述語論理の推論}
  \begin{theo}[5.81 量化の影響がない場合]
    \(\xi \not \in \textit{fv}(\phi)\)のとき
    \[
    \forall \xi \phi \Dashv \vDash \phi, \exists \xi \phi \Dashv \vDash \phi
    \]
  \end{theo}
  \begin{block}{練習問題 5.82}
    定理~5.81は補題~5.59(2)より明らかであるが,定理~5.71--定理~5.74のみを用いた証明を考えよ.
  \end{block}
  \begin{block}{証明}
    \(\forall \xi \phi \Dashv \vDash \phi\)について
    \[
    \begin{array}{rrcll}
      (\Rightarrow_1) & \phi [ \xi / \xi ] & \vDash & \phi & (\textit{定理5.42(1)と同一律}) \\
      (\Rightarrow_2) & \forall \xi \phi & \vDash & \phi & ((\Rightarrow_1)\textit{と全称除去律}) \\
      (\Leftarrow_1) & \phi & \vDash & \phi [ \xi / \xi ] & (\textit{定理5.42(1)と同一律}) \\
      (\Leftarrow_2) & \phi & \vDash & \forall \xi \phi & ((\Leftarrow_1)\textit{と全称導入律}, \xi \not \in \textit{fv}(\phi), \xi \not \in \textit{fv}(\forall\xi\phi)) \\
    \end{array}
    \]
  \end{block}
\end{frame}

\begin{frame}{5.4.4 一階述語論理の推論}
  \begin{block}{証明 (cont.)}
    \(\exists \xi \phi \Dashv \vDash \phi\)について
    \[
    \begin{array}{rrcll}
      (\Rightarrow_1) & \phi [ \xi / \xi ] & \vDash & \phi & (\textit{定理5.42(1)と同一律}) \\
      (\Rightarrow_2) & \exists \xi \phi & \vDash & \phi & ((\Rightarrow_1)\textit{と存在除去律}, \xi \not \in \textit{fv}(\phi), \xi \not \in \textit{fv}(\exists\xi\phi)) \\
      (\Leftarrow_1) & \phi & \vDash & \phi [ \xi / \xi ] & (\textit{定理5.42(1)と同一律}) \\
      (\Leftarrow_2) & \phi & \vDash & \exists \xi \phi & ((\Leftarrow_1)\textit{と存在導入律}) \\
    \end{array}
    \]
  \end{block}
\end{frame}

\begin{frame}{5.4.4 一階述語論理の推論}
  \begin{theo}[5.83 \(\forall\)と\(\exists\)が相互に定義可能であること]
    \[
    \neg\forall\xi\phi \Dashv\vDash \exists\xi(\neg\phi),
    \neg\exists\xi\phi \Dashv\vDash \forall\xi(\neg\phi)
    \]
  \end{theo}
  \begin{block}{左式の証明}
    \(\Rightarrow\)
    \[
    \begin{array}{rrcll}
      (1) & \neg\phi[\zeta/\xi] & \vDash & \neg\phi[\zeta/\xi] & \textit{同一律} \\
      (2) & \neg\phi[\zeta/\xi] & \vDash & \exists\xi(\neg\phi) & \textit{存在導入律} \\
      (3) & \neg\exists\xi(\neg\phi), \neg\phi[\zeta/\xi] & \vDash & \bot & \textit{2と背理法 p.54} \\
      (4) & \neg\exists\xi(\neg\phi) & \vDash & \phi[\zeta/\xi] & \textit{3と背理法} \\
      (5) & \neg\exists\xi(\neg\phi) & \vDash & \forall\xi\phi & \zeta\not\in\textit{fv}(\neg\exists\xi(\neg\phi))\cup\textit{fv}(\forall\xi\phi) \\
      & & & & \textit{と全称導入律} \\
      (6) & \neg \forall \xi \phi, \neg\exists\xi(\neg\phi) & \vDash & \bot & \textit{5と背理法} \\
      (7) & \neg \forall \xi \phi & \vDash & \exists \xi (\neg \phi) & \textit{6と背理法} \\
    \end{array}
    \]
  \end{block}
\end{frame}

\begin{frame}{5.4.4 一階述語論理の推論}
  \begin{block}{左式の証明 cont.}
    \(\Leftarrow\)
    \[
    \begin{array}{rrcll}
      (1) & \neg\neg\forall\xi\phi & \vDash & \forall\xi\phi & \textit{二重否定除去律} \\
      (2) & \forall\xi\phi & \vDash & \phi[\zeta/\xi] & \textit{全称例化律} \\
      (3) & \phi[\zeta/\xi],\neg\phi[\zeta/\xi] & \vDash & \bot & \textit{恒真式}\phi\wedge\neg\phi\leftrightarrow\bot \\
      (4) & \neg\neg\forall\xi\phi,\neg\phi[\zeta/\xi] & \vDash & \bot & \textit{1,2,3とカット p.52}\\
      (5) & \neg\phi[\zeta/\xi] & \vDash & \neg\forall\xi\phi & \textit{4と背理法} \\
      (6) & \exists\xi(\neg\phi) & \vDash & \neg\forall\xi\phi & \zeta\not\in\textit{fv}(\exists\xi(\neg\phi))\cup\textit{fv}(\neg\forall\xi\phi) \\
      & & & & \textit{と存在除去律} \\
    \end{array}
    \]
  \end{block}
\end{frame}

\begin{frame}{5.4.4 一階述語論理の推論}
  \begin{block}{練習問題~5.84 右式の証明}
    \(\Rightarrow\)
    \[
    \begin{array}{rrcll}
      (1) & \neg\neg\phi[\zeta/\xi] & \vDash & \phi[\zeta/\xi] & \textit{二重否定除去律} \\
      (2) & \phi[\zeta/\xi] & \vDash & \exists\xi\phi & \textit{存在汎化律(ただし\(\zeta\not\in\phi\))} \\
      (3) & \exists\xi\phi, \neg\exists\xi\phi & \vDash & \bot & \textit{恒真式}\phi\wedge\neg\phi\leftrightarrow\bot \\
      (4) & \neg\neg\phi[\zeta/\xi], \neg\exists\xi\phi & \vDash & \bot & \textit{1,2,3とカット} \\
      (5) & \neg\exists\xi\phi & \vDash & \neg\phi[\zeta/\xi] & \textit{背理法} \\
      (6) & \neg\exists\xi\phi & \vDash & \forall\xi(\neg\phi) & \zeta\not\in\textit{fv}(\neg\exists\xi\phi)\cup\textit{fv}(\neg\phi) \\
      & & & & \textit{と全称導入律}
    \end{array}
    \]
  \end{block}
\end{frame}

\begin{frame}{5.4.4 一階述語論理の推論}
  \begin{block}{練習問題~5.84 右式の証明 cont.}
    \(\Leftarrow\)
    \[
    \begin{array}{rrcll}
      (1) & \forall\xi(\neg\phi) & \vDash & \forall\xi(\neg\phi) & \textit{同一律} \\
      (2) & \forall\xi(\neg\phi) & \vDash & \neg\phi[\zeta/\xi] & \textit{全称例化律(ただし\(\zeta\not\in\phi\))} \\
      (3) & \forall\xi(\neg\phi), \phi[\zeta/\xi] & \vDash & \bot & \textit{2と背理法} \\
      (4) & \phi[\zeta/\xi] & \vDash & \neg\forall\xi(\neg\phi) & \textit{3と背理法} \\
      (5) & \exists\xi\phi & \vDash & \neg\forall\xi(\neg\phi) & \zeta\not\in\textit{fv}(\exists\xi\phi)\cup\textit{fv}(\neg\forall\xi(\neg\phi)) \\
      & & & & \textit{と存在除去律} \\
      (6) & \forall\xi(\neg\phi), \exists\xi\phi & \vDash & \bot & \textit{5と背理法} \\
      (7) & \forall\xi(\neg\phi) & \vDash & \neg\exists\xi\phi & \textit{6と背理法} \\
    \end{array}
    \]
  \end{block}
\end{frame}

\begin{frame}{5.4.4 一階述語論理の推論}
  \begin{theo}[5.85]
    \(\xi\not\in\textit{fv}(\phi)\)のとき以下が成り立つ
    \begin{eqnarray*}
      \phi\wedge\forall\xi\psi & \Dashv\vDash & \forall\xi(\phi\wedge\psi) \\
      \phi\wedge\exists\xi\psi & \Dashv\vDash & \exists\xi(\phi\wedge\psi) \\
      \phi\vee\forall\xi\psi & \Dashv\vDash & \forall\xi(\phi\vee\psi) \\
      \phi\vee\exists\xi\psi & \Dashv\vDash & \exists\xi(\phi\vee\psi)
    \end{eqnarray*}
  \end{theo}
\end{frame}

\begin{frame}{5.4.4 一階述語論理の推論}
  \begin{block}{練習問題~5.86}
    定理~5.85の証明 (1)
    \[
    \begin{array}{rll}
      & \llbracket \forall \xi (\phi \wedge \psi) \rrbracket_{M,g} = 1 & \\
      \iff & \textit{任意の}a \in D_M\textit{について} \llbracket \phi \wedge \psi \rrbracket_{M,g[a\mapsto a]} = 1 & \\
      \iff & \textit{任意の}a \in D_M\textit{について} \llbracket \phi \rrbracket_{M,g[\xi\mapsto a]} = 1 \textit{かつ} \llbracket \psi \rrbracket_{M,g[\xi\mapsto a]} = 1 & \\
      \iff & \textit{任意の}a \in D_M\textit{について} \llbracket \phi \rrbracket_{M,g} = 1 \textit{かつ} \llbracket \psi \rrbracket_{M,g[\xi\mapsto a]} = 1 & \\
      \iff & \llbracket \phi \rrbracket_{M,g} = 1 \textit{かつ} \textit{任意の}a \in D_M\textit{について} \llbracket \psi \rrbracket_{M,g[\xi\mapsto a]} = 1 & \\
      \iff & \llbracket \phi \rrbracket_{M,g} = 1 \textit{かつ} \llbracket \forall \xi \psi \rrbracket_{M,g} = 1 & \\
      \iff & \llbracket \phi \wedge \forall \xi \psi \rrbracket_{M,g} = 1 & \\
    \end{array}
    \]
  \end{block}

  4行目から5行目があやしい→ホワイトボードで別証
\end{frame}

\begin{frame}{5.4.5 \(\alpha\)同値性}
  \begin{theo}[5.87 (\(\alpha\)同値性)]
    \(\zeta\not\in\textit{fv}(\phi)\)のとき
    \begin{eqnarray*}
      \forall \xi \phi & \Dashv\vDash & \forall \zeta ( \phi [ \zeta / \xi ] ) \\
      \exists \xi \phi & \Dashv\vDash & \exists \zeta ( \phi [ \zeta / \xi ] )
    \end{eqnarray*}
  \end{theo}

  \[
  \begin{array}{rll}
    & \llbracket \phi [ \zeta / \xi ] \rrbracket_{M,g[\zeta\mapsto a]} & \\
    = & \llbracket \phi \rrbracket_{M,g[\zeta\mapsto a][\xi\mapsto\llbracket\zeta\rrbracket_{M,g[\zeta\mapsto a]}]} & \\
    = & \llbracket \phi \rrbracket_{M,g[\zeta\mapsto a][\xi\mapsto a]} & \\
    = & \llbracket \phi \rrbracket_{M,g[\xi\mapsto a][\zeta\mapsto a]} & \\
    = & \llbracket \phi \rrbracket_{M,g[\xi\mapsto a]} & 
  \end{array}
  \]

  \[
  \begin{array}{rll}
    & \llbracket \forall \phi \rrbracket_{M,g} = 1 & \\
    \iff & \textit{すべての}a\in D_M\textit{について}\llbracket\phi\rrbracket_{M,g[\zeta\mapsto a]} = 1 & \\
    \iff & \textit{すべての}a\in D_M\textit{について}\llbracket\phi[\zeta/\xi]\rrbracket_{M,g[\zeta\mapsto a]} = 1 & \\
    \iff & \llbracket\forall\zeta(\phi[\zeta/\xi])\rrbracket_{M,g[\zeta\mapsto a]} = 1 &

  \end{array}
  \]

\end{frame}

\begin{frame}{5.4.6 置き換え}
  \begin{itemize}
  \item \predl{}の論理式の置き換え操作は\propl{}でも可能
  \item \propl{}では量化論理式について考える必要がある
  \end{itemize}

  \begin{theo}[5.89]
    \begin{eqnarray*}
      \phi \Dashv \vDash \psi & \Rightarrow & \forall \xi \phi \Dashv \vDash \forall \xi \psi \\
      \phi \Dashv \vDash \psi & \Rightarrow & \exists \xi \phi \Dashv \vDash \exists \xi \psi \\
    \end{eqnarray*}
  \end{theo}

\end{frame}

\begin{frame}{5.4.6 置き換え}
\end{frame}

\begin{frame}{5.4.6 置き換え}
\end{frame}

\end{document}

